% generated by GAPDoc2LaTeX from XML source (Frank Luebeck)
\documentclass[a4paper,11pt]{report}

\usepackage{a4wide}
\sloppy
\pagestyle{myheadings}
\usepackage{amssymb}
\usepackage[utf8]{inputenc}
\usepackage{makeidx}
\makeindex
\usepackage{color}
\definecolor{FireBrick}{rgb}{0.5812,0.0074,0.0083}
\definecolor{RoyalBlue}{rgb}{0.0236,0.0894,0.6179}
\definecolor{RoyalGreen}{rgb}{0.0236,0.6179,0.0894}
\definecolor{RoyalRed}{rgb}{0.6179,0.0236,0.0894}
\definecolor{LightBlue}{rgb}{0.8544,0.9511,1.0000}
\definecolor{Black}{rgb}{0.0,0.0,0.0}

\definecolor{linkColor}{rgb}{0.0,0.0,0.554}
\definecolor{citeColor}{rgb}{0.0,0.0,0.554}
\definecolor{fileColor}{rgb}{0.0,0.0,0.554}
\definecolor{urlColor}{rgb}{0.0,0.0,0.554}
\definecolor{promptColor}{rgb}{0.0,0.0,0.589}
\definecolor{brkpromptColor}{rgb}{0.589,0.0,0.0}
\definecolor{gapinputColor}{rgb}{0.589,0.0,0.0}
\definecolor{gapoutputColor}{rgb}{0.0,0.0,0.0}

%%  for a long time these were red and blue by default,
%%  now black, but keep variables to overwrite
\definecolor{FuncColor}{rgb}{0.0,0.0,0.0}
%% strange name because of pdflatex bug:
\definecolor{Chapter }{rgb}{0.0,0.0,0.0}
\definecolor{DarkOlive}{rgb}{0.1047,0.2412,0.0064}


\usepackage{fancyvrb}

\usepackage{mathptmx,helvet}
\usepackage[T1]{fontenc}
\usepackage{textcomp}


\usepackage[
            pdftex=true,
            bookmarks=true,        
            a4paper=true,
            pdftitle={Written with GAPDoc},
            pdfcreator={LaTeX with hyperref package / GAPDoc},
            colorlinks=true,
            backref=page,
            breaklinks=true,
            linkcolor=linkColor,
            citecolor=citeColor,
            filecolor=fileColor,
            urlcolor=urlColor,
            pdfpagemode={UseNone}, 
           ]{hyperref}

\newcommand{\maintitlesize}{\fontsize{50}{55}\selectfont}

% write page numbers to a .pnr log file for online help
\newwrite\pagenrlog
\immediate\openout\pagenrlog =\jobname.pnr
\immediate\write\pagenrlog{PAGENRS := [}
\newcommand{\logpage}[1]{\protect\write\pagenrlog{#1, \thepage,}}
%% were never documented, give conflicts with some additional packages

\newcommand{\GAP}{\textsf{GAP}}

%% nicer description environments, allows long labels
\usepackage{enumitem}
\setdescription{style=nextline}

%% depth of toc
\setcounter{tocdepth}{1}





%% command for ColorPrompt style examples
\newcommand{\gapprompt}[1]{\color{promptColor}{\bfseries #1}}
\newcommand{\gapbrkprompt}[1]{\color{brkpromptColor}{\bfseries #1}}
\newcommand{\gapinput}[1]{\color{gapinputColor}{#1}}


\begin{document}

\logpage{[ 0, 0, 0 ]}
\begin{titlepage}
\mbox{}\vfill

\begin{center}{\maintitlesize \textbf{ symchm \mbox{}}}\\
\vfill

\hypersetup{pdftitle= symchm }
\markright{\scriptsize \mbox{}\hfill  symchm  \hfill\mbox{}}
{\Huge \textbf{ A package for symmetric polyhedral projection \mbox{}}}\\
\vfill

{\Huge  1.0 \mbox{}}\\[1cm]
{ 05/11/2016 \mbox{}}\\[1cm]
\mbox{}\\[2cm]
{\Large \textbf{ Jayant Apte\\
    \mbox{}}}\\
\hypersetup{pdfauthor= Jayant Apte\\
    }
\end{center}\vfill

\mbox{}\\
{\mbox{}\\
\small \noindent \textbf{ Jayant Apte\\
    }  Email: \href{mailto://jayant91089@gmail.com} {\texttt{jayant91089@gmail.com}}\\
  Homepage: \href{https://sites.google.com/site/jayantapteshomepage/} {\texttt{https://sites.google.com/site/jayantapteshomepage/}}\\
  Address: \begin{minipage}[t]{8cm}\noindent
 Department of Electrical and Computer Engineering\\
 Drexel University\\
 Philadelphia, PA 19104\\
 \end{minipage}
}\\
\end{titlepage}

\newpage\setcounter{page}{2}
\newpage

\def\contentsname{Contents\logpage{[ 0, 0, 1 ]}}

\tableofcontents
\newpage

 \index{\textsf{symchm}}     
\chapter{\textcolor{Chapter }{Introduction}}\label{Chapter_Introduction}
\logpage{[ 1, 0, 0 ]}
\hyperdef{L}{X7DFB63A97E67C0A1}{}
{
  symchm is a GAP package that for projeting polyhedra using Convex Hull Method
(chm). The 'sym' prefix follows from the fact that symchm also supports
specifying a group of symmetries of the projection polyhedron. Currently, the
main supported class of symmetries is the permutations of co-ordinate
dimensions under which the projection is known to be fixed (stabilized
setwise). The algorithm CHM proceeds by solving a series of linear programs
(LPs) over the input polytope $P$, recovering a vertex of projection per LP solved. It also maintains an
inequality description of an inner bound of projection, associated with the
convex hull of the subset of vertices found. This description is updated every
time a new vertex is found. symchm exploits symmetry in several different ways
viz. by solving roughly the number of LPs equal to the number orbits of the
symmetry group on vertices of projection and by using symmetric updates of the
inequality description. The aforementioned LPs are solved by an external
program Qsopt$\_$ex \cite{qs} which is a linear program solver by Applegate,Cook,Dash and Espinoza. symchm
uses GAP interface package $\texttt{qsopt}\_\texttt{ex-interface}$ \cite{jayantqsint} to talk to Qsopt$\_$ex via standard input-output. }

   
\chapter{\textcolor{Chapter }{Installation}}\label{Chapter_Installation}
\logpage{[ 2, 0, 0 ]}
\hyperdef{L}{X8360C04082558A12}{}
{
  Assuming you already have GAP 4.5+ installed, you can follow the steps below
to install the package: 
\begin{itemize}
\item  To get the newest version of symchm, download the .zip archive from \href{https://github.com/jayant91089/symchm} {\texttt{https://github.com/jayant91089/symchm}} and unpack it using $\texttt{unzip symchm-x.y.zip}$ in the terminal. Do this preferably inside the $pkg$ subdirectory of your GAP 4 installation. It creates a subdirectory called $\texttt{qsopt}\_\texttt{ex-interface}$. If you do not know the whereabouts of the $pkg$ subdirectory, invoke the following in GAP: 
\end{itemize}
 
\begin{Verbatim}[fontsize=\small,frame=single,label=Code]
  GAPInfo.("RootPaths");
\end{Verbatim}
 Look for pkg directory inside any of the paths returned. 
\begin{itemize}
\item  Once unpacked in the right directory, one can start using symchm by invoking 
\end{itemize}
 
\begin{Verbatim}[fontsize=\small,frame=single,label=Code]
  LoadPackage( "symchm");
\end{Verbatim}
 from within GAP. This will automatically load $\texttt{qsopt}\_\texttt{ex-interface}$, if it is available. If instead, it returns 'fail', make sure $\texttt{qsopt}\_\texttt{ex-interface}$ is installed. See the package manual for $\texttt{qsopt}\_\texttt{ex-interface}$ for details of how to install it. }

 \def\bibname{References\logpage{[ "Bib", 0, 0 ]}
\hyperdef{L}{X7A6F98FD85F02BFE}{}
}

\bibliographystyle{alpha}
\bibliography{symchm.bib}

\addcontentsline{toc}{chapter}{References}

\def\indexname{Index\logpage{[ "Ind", 0, 0 ]}
\hyperdef{L}{X83A0356F839C696F}{}
}

\cleardoublepage
\phantomsection
\addcontentsline{toc}{chapter}{Index}


\printindex

\newpage
\immediate\write\pagenrlog{["End"], \arabic{page}];}
\immediate\closeout\pagenrlog
\end{document}
